\subsection{Introducción}\label{header-n2}

Generalmente, dada la complejidad que implican los códigos que resuelven
la dinámica de los flujos en OpenFOAM, se suele tomar como referencia un
caso y aplicar los diferentes cambios al mismo.

En un comienzo se parte por el caso "damBreak" (tutorial de OpenFOAM),
ya que, por su naturaleza, se asemeja a las condiciones que se podrían
ensayar. Además, se simularon diferentes ejemplos de casos, que
sirvieron como orientación para añadir nuevas funciones y, así, extraer
información adicional. Cada implementación se fue ejecutando por
separado para comprobar la ausencia de errores.

Además, cabe mencionar que se trató de implementar las condiciones de
contorno para la generación del oleaje a través de las librerías del
código ihFOAM. No obstante, dada la complejidad que conlleva su
definición tanto en el laboratório como en el modelo, se consideró más
apropiado dejarlo para futuros proyectos. Actualmente, estas
condiciones, también se han implementado en las versiones de OpenFOAM.\\

Por otra parte, una vez obtenida la solución aproximada del modelo por
ordenador, se fue adaptando el ensayo hasta llegar a la solución final,
lo que implicó, nuevos cambios en el caso. En este apartado se describen
y analizan los casos que sirvieron de guía antes de llegar a la solución
final, dada en los apartados posteriores /R/XX CANAL 2-D y XX CANAL 3-D.

\subsection{Lista de casos}\label{header-n13}

\begin{itemize}
\item
  \textbf{damBreak}: Tutorial de OpenFOAM ubicado en
  \texttt{\textless{}\$FOAM\_RUN/tutorials/multiphase/interFoam/\textgreater{}}.
  El caso trata modelos multifásicos, 2-D, incompresible, transitorio,
  con y sin turbulencia. Además, utiliza el código interFoam que
  resuelve las ecuaciones de continuidad y, usando el método VOF, se
  calcula la interfase. Completamente explicado en
  \href{http://cfd.direct/openfoam/user-guide/dambreak/}{User Guide: 2.3
  Breaking of a dam}.
\item
  \textbf{damBreakFine}: Incremento de la resolución del mallado del
  caso anterior. Así mismo, se ejecuta en paralelo siguiendo las
  explicasiones contenidas en
  \href{http://cfd.direct/openfoam/user-guide/running-applications-parallel}{User
  Guide: 3.4 Running applications in parallel}.
\item
  \textbf{damBreakMod}: Primera aproximación al modelo OWC, variando la
  geometría del caso anterior desde el fichero \emph{blockMeshDict}
  incluyendo una cavidad que haga de cámara.
\item
  \textbf{damBreakSnappy}: Tal y como se ha mencionado OpenFOAM
  proporciona el diccionario \emph{snappyHexMeshDict}, para generar
  mallas de alta calidad a partir de un modelo generado desde un
  programa CAD. En este caso, se genera el modelo de forma paramétrica
  desde OpenSCAD, se exporta a STL, dejando la descomposición mínima
  compuesta por triángulos (para los cilindros se especifica el número
  de frames) ya que, luego se utilizará snappyHexMesh para el refinado.
  Se definen los contornos desde Blender
  (\href{https://github.com/nogenmyr/swiftBlock}{GitHub/nogenmyr}),
  exportando cada uno por separado en ascii.stl y se juntan después en
  un solo fichero. Se modifica el diccionario snappyHexMeshDict dejando
  los campos mínimos necesarios para realizar un mallado básico.
\item
  \textbf{damBreak3d}: Caso damBreak en 3-D, modificando la geometría en
  el eje \emph{z} y definiendo el contorno de "frontAndBack" de forma
  apropiada para cada variable del caso. Adicionalmente, se puede
  modificar el fichero \texttt{setFields} para variar el volumen de la
  columna de agua inicial. Se toma de referencia las explicaciones de
  \href{http://www.calumdouglas.ch/openfoam-example-3d-dambreak/}{Calum
  Duglas "damBreak example modified to 3d"}.
\item
  \textbf{canal2D}: Modelo en 2D del canal del laboratorio con la pared
  que delimita la cámara y una ranura en la chimenea de salida
  (\(19 mm\) en \emph{z}). Del mismo se obtiene la presión del flujo de
  aire aguas arriba del diafragma y el caudal de salida, teniendo como
  referencia los tutoriales:

  \begin{itemize}
  \item
    \texttt{simpleFoam/pitzDaily}: a partir del cual se determina cómo
    obtener los residuos, el valor de las variables de interés en puntos
    conctretos (\emph{probes}) o en los centros de cada celda contenidas
    entre dos puntos (\emph{sample}).
  \item
    \texttt{multiphase/interFoam/ras/waterChannel}: donde se halla el
    flujo de salida (\emph{outletFlux}) en la superficie especificada
    por el usuario mediante una función definida en el diccionario
    "\emph{controlDict}".
  \item
    \texttt{multiphase/potentialFreeSurfaceFoam/oscillatingBox}:
    mediante el cual se añaden paredes al modelo con la función
    \emph{topoSet}, permitiendo realizar modificaciones parciales en la
    geometría (generada a partir de blockMesh).
  \end{itemize}

  Ya que el cálculo del flujo de salida se resulve en función del área
  de la sección que atraviesa, se realiza otro caso considerando la
  anchura del canal (\(80 mm\) en \emph{z}) resultando en una importante
  variación en el valor del caudal. Por ello, la aproximación en 2-D se
  considera apropiada para visualizar el comportamiento de los flujos,
  pero insuficiente para tomar como válidos los valores obtenidos y
  poder compararlos con los experimentales. \\
\item
  \textbf{canal3D}: Simulación del caso del laboratorio, realizando las
  modificaciones necesarias para aproximar el caso a lo que se
  experimentará en la realidad. El modelo se genera a partir del
  programa \emph{Salome}, se modifican algunos valores de las
  propiedades físicas y los resultados se procesan desde \emph{ParaView}
  a partir de \emph{scripts} escritos en \emph{Python}. Estos extraen el
  valor medio de la altura del agua en la cámara y la presión
  manométrica en el mismo punto que en el ensayo, a lo largo del tiempo.
  Además se obtiene el flujo de salida a través del diafragma, del mismo
  modo que para el caso en 2-D.
\end{itemize}

\subsection{Ejecución automatizada}\label{header-n52}

Los casos de ejemplo de OpenFOAM, se pueden ejecutar de forma
automatizada mediante \emph{scripts}, un \emph{script} es un archivo
donde se listan los pasos de la ejecución, es decir, son una serie de
órdenes que se irán ejecutando una a una. A este archivo se le añaden
permisos de ejecución para que el sistema los reconozca como
\emph{scripts} o archivos ejecutables (\texttt{chmod\ +x\ filename}).

Teniendo de referencia los \emph{scripts} diseñados para los casos de
ejemplo implicados, se aunan en uno sólo, de forma que se describan las
ordenes para cada caso estudiado en este proyecto, desde un solo
fichero. También se incluye el \emph{script} para borrar el caso y
dejarlo como al comienzo, la lista de casos se añade al final del
\emph{script} y la forma de ejecutarlos sería la siguiente:

\begin{itemize}
\item
  \texttt{./RunCases.sh\ CASE}
\item
  \texttt{./CleanCase.sh\ CASE}
\end{itemize}

Estas órdenes se pueden utilizar, si se dispone de las librerías
necesarias de OpenFOAM instaladas. La mayoría de casos se ejecutan en un
portátil con 4 procesadores Intel Core i7-4510U, y con el sistema
operativo Fedora 24. También se usa el ordenador del laboratorio con
windows 7 (Intel Xeon E5-2630 con 8 procesadores de doble núcleo y 16GB
de RAM), luego se emplea una máquina virtual con Ubuntu14. Además,
varias pruebas del caso en 3-D se ejecutan en un ordenador de mesa
(Intel Core i3-8350K de 4 procesadores y 16GB de RAM) con windows10
mediante la herramienta de \emph{Docker}, la cual no necesita la
instalación del programa.
