Los convertidores OWCs son dispositivos que transforman la energía del
oleaje en energía útil; tienen que ser capaces de resistir los embates
del mar y de funcionar eficientemente para las amplias gamas de
frecuencia y amplitud de las olas.

Los primeros testimonios sobre la utilización de la energía de las olas
se encuentran en China, en donde en el siglo XII empiezan a operar
molinos por acción del oleaje. Al principio de este siglo, el francés
Bouchaux-Pacei suministra electricidad a su casa en Royan, mediante un
sistema neumático, parecido a las actuales columnas oscilantes. En esta
misma época se prueban sistemas mecánicos en California, y en 1920 se
ensaya un motor de péndulo en Japón.
\href{http://files.pfernandezdiez.es/EnergiasAlternativas/mar/PDFs/02Olas.pdf}{P.Fernández}

Uno de los pioneros en el campo del aprovechamiento de la energía de las
olas fue el japonés Yoshio Masuda. Empezó sus investigaciones en 1945,
ensayó en el mar en 1947, el primer prototipo de una balsa. A partir de
1960 desarrolla un sistema neumático para la carga de baterías en boyas
de navegación, con una turbina de aire de 60 W, de la que se vendieron
más de 1200 unidades. El prototipo OWC se denomina así tras su inventor
Yoshio Masuda.

La central de La Rance, en Francia, se puso en funcionamiento en 1967,
obteniendo una generación anual de 4400GWh y una potencia instalada de
240MW.

En 1973 con la crisis del aceite, algunos países no podían permitirse
comprar la energía a otros, con lo que la energía del mar experimentó un
hauge en sus investigacione, probando nuevas formas de obtención de
energía.

La investigación a gran escala del aprovechamiento de la energía de las
olas se inicia a partir de 1974 en varios centros del Reino Unido,
estudiándose sofisticados sistemas para grandes aprovechamientos,
actividad que se abandona casi totalmente en 1982, por falta de recursos
económicos. A mediados de los ochenta entran en servicio varias plantas
piloto de distintos tipos en Europa y Japón.

El primer convertidor se patentó en Francia en 1799 Girard e hijo, con
la construcción de una balsa unida a una polea en tierra (tipo boya).

El verdadero desarrollo comienza en el último cuarto de siglo XX. Cuando
se abren varias empresas para tratar de competir en el mercado
eléctrico, entre otras: Aqua Energy Group (USA), Archimedes Wave Swing
(Paises Bajos), Emergetech (Australia), Ocean Power Delivery (Reino
Unido), Wavegen (EEUU), Wave Plane International (Dinamarca), Wavemill
Energy (Canada).

A principios del siglo XX Bouchaux-Pacer suministra electricidad a su
casa en Royan. Asimismo, en 1921 el Instituto Oceanográfico de Monaco
utiliza una bomba accionada por las olas para elevar agua a 60m con una
potencia de 400W.

Los desarrollos posteriores se analizan con más detalle en el apartado
/R/{[}Alternativas{]}.
