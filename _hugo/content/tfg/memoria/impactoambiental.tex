En este apartado, se pretende dar unas consideradciones generales de los
efectos sobre el medio ambiente que dependerán estrechamente del tipo de
dispositivo diseñado. Entre los posibles impactos negativos, pueden
señalarse los siguientes:

\begin{enumerate}
\def\labelenumi{\arabic{enumi}.}
\item
  \textbf{Impacto sobre la vida del litoral} Durante el proceso de
  instalación de captadores de energía undimotriz, ubicados frente al
  espigón, puede ser necesaria la contrucción de una planta de
  producción y encofrado de hormigón. Por lo que debe estudiarse la
  posible modificación o destrucción temporal del hábitat, a fin de
  minimizarlo.
\item
  \textbf{Impacto sobre la vida marina}

  Este tipo de impacto puede extenderse a todas las zonas del litoral,
  supralitoral, mesolitoral y sublitoral, además ha de considerarse el
  impacto directo y permanente del colector en la ubicación determinada.
  Ya que, se puede llegar a destruir, a nivel local, el hábitat de
  especies tanto vegetales como animales.

  Los captadores de energía mareomotriz a gran escala, son los que más
  impacto generan sobre la vida marina, ya que se modifica el flujo de
  corriente habitual, reteniendo el volumen del agua causado por la
  pleamar, para luego hacer pasar ese agua por unas turbinas.

  Sin embargo, los captadores de energía undimotriz, suelen ser de menor
  tamaño, adaptados en diques, junto al espigón existente o en mar
  abierto, de tal forma que se aproveche la energia sin modificar el
  mediose pueda considerar un impacto mínimo. Es más, en instalaciónes
  como la de Port Kembla (Australia), la estructura sirvió para el
  habitat de especies marinas. 
\item
  \textbf{Impacto sobre la morfología del litoral}

  Debido a las variaciones en la morfología del litoral, se producen
  modificaciones en los mecanismos de sedimentación.

  No obstanter cuando, los colectores de energía undimotriz están
  montados sobre o junto a estructras ya construidas como diques, luego
  no se produce una modificación del litoral adicional.
\item
  \textbf{Otros impactos}

  \begin{itemize}
  \item
    \textbf{Impacto visual}

    El impacto visual depende del tipo de aparato y de su distancia de
    la línea de costa. En general, un sistema de boya flotante o una
    plataforma situada mar a dentro o un sistema sumergido,
    probablemente no presente mucho impacto visual. Cuando un área
    depende del turismo, la obstrucción visual es crítica.
  \item
    \textbf{Impacto sonoro}:

    Los sistemas de conversión de la energía de las olas producen ruido,
    aunque los niveles suelen ser menores que los ruidos de un barco.
    Cuando operan a plena carga, no se espera que sean más ruidosos que
    el viento o las olas, además, estos sistemas pueden ser construidos
    con un buen material aislador de ruidos.

    El ruido generado puede viajar largas distancias debajo del agua y
    pueden afectar a ciertos animales, tales como las ballenas, las
    focas, etc. Se precisan más investigaciones para determinar si
    existen impactos en la vida de los mamíferos debido al ruido de
    estos aparatos.
  \item
    \textbf{Impacto sobre las actividades humanas}:

    La instalación de los colectores en el litoral, puede limitar en
    cierto modo el acceso al agua, hecho que puede interferir a ciertas
    actividades como la pesca. Otros tipos de colectores, ubicados en
    altamar, podrían interferir con las rutas de los barcos, por lo que
    habrá que señalizarlos correctamente, aún así no presentarían
    grandes alteraciones dadas sus dimensiones.
  \end{itemize}
\end{enumerate}
