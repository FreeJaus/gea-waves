\subsection{Introducción}\label{header-n2}

Con este trabajo se pretende realizar un análisis de las tecnologías
existentes para el aprovechamiento del mar y, junto con ello, estudiar
un caso práctico mediante un ensayo y unas simulaciones por ordenador
para comparar los resultados. Adicionalmente, se tratará de reunir la
información más relevante que analice las consideraciones a tener en
cuenta para simular un caso de principio a fin, facilitando al ususario
entender los pasos a seguir para simular un fluido o varios, en unas
condiciones determinadas, partiendo de un diseño modelado por ordenador
(CAD, Computer-aided design).

Las técnicas CFD son ya, a todos los efectos, una herramienta más dentro
de la ingeniería asistida por ordenador (CAE, Computer-aided
engineering), utilizada universalmente en la industria. Sus
posibilidades para simular todo tipo de fenómenos y flujos hace que los
softwares CFD sean una parte indispensable en procesos de diseño
aerodinámico, hidrodinámico o procesos productivos.

No obstante, las tecnologías para el aprovechamiento del mar, aún se
encuentran en etapa de desarrollo, por lo que no cuenta con la totalidad
de la información para su instalación, operación y mantenimiento, por
ello será necesaria la aplicación de suposiciones. Aun así, las
investigaciones desarrolladas en este campo, con la aparición de
numerosos prototipos, hacen de esta tecnología una de las más
prometedora en cuanto a su rendimiento y funcionalidad. En comparación
con otras energías renovables, el momento y la intensidad de las mareas
y corrientes se puede predecir con siglos de anticipación,
/R/\href{Romero\%20García,RE;\%20\%22Producción\%20de\%20energía\%20eléctrica\%20a\%20partir\%20de\%20los\%20mares\%22;\%20Técnica\%20Industrial;\%20288;\%20Agosto\%202010;\%20pp\%2044-51}{Romero
García, RE}. Además, su densidad es muy superior a la del aire, con lo
que, de la energía de las olas se obtienen altos potenciales.

\subsection{Actividades y procedimientos}\label{header-n9}

\begin{itemize}
\item
  Estudio del estado del arte:

  \begin{itemize}
  \item
    Perspectiva histórica y estado actual de la energía del mar
    /R/\href{Montero\%20Sousa,\%20JA\%20y\%20Calvo\%20Rolle,\%20JL;\%20\%22Energía\%20mareomotriz:\%20perspectiva\%20histórica\%20y\%20estado\%20actual\%22;\%20Técnica\%20Industrial;\%20301;\%20marzo2013;\%20pp54-60}{Montero,
    JA et al}, caracterizando esta fuente de energía y los avances
    logrados hasta el momento.
  \item
    Clasificación y descripción del funcionamiento de los dispositivos
    deseñados para la obtención de energía
    /R/\href{\%22Energía\%20del\%20oleaje\%22;\%20Eduambiental;\%20pp\%20515-550.\%20http://comunidad.eduambiental.org/file.php/1/curso/contenidos/docpdf/capitulo22.pdf}{Eduambiental:
    Energía del oleaje}.
  \item
    Evaluación del estado actual del desarrollo de las tecnologías más
    prometedoras.
  \item
    Principales iniciativas europeas de parques de olas y zonas de
    prueba de prototipos a escala.
  \item
    Nivel energético disponible en la costa española.
  \item
    Caracterización de las olas
    /R/\href{Pelissero,\%20M\%20et\%20al.;\%20\%22Aprovechamiento\%20de\%20la\%20Energía\%20Undimotriz\%22;\%20Proyecciones,\%20Vol\%209\%20No.\%202,\%20octubre\%202011;\%20pp\%2053-65}{Pelissero,M
    et al}, generación, tipos de ondas y ecuaciones que gobiernan la
    teoría de las olas.
  \end{itemize}
\item
  Modelo físico:

  \begin{itemize}
  \item
    Definición de las condiciones iniciales del caso, identificando los
    fenómenos físicos que pretenden modelarse.
  \item
    Análisis teórico de las ecuaciones que intervendran en la resolución
    del problema.
  \item
    Métodos utilizados para resolver sistemas complejos.
  \end{itemize}
\item
  Simulación por ordenador:

  \begin{itemize}
  \item
    Análisis de las herramientas disponibles, seleccionando el más
    conveniente para el trabajo que se desea realizar.
  \item
    Comprender cómo funciona OpenFOAM, realizando diferentes casos
    descritos.
  \item
    Definición del modelo geométrico que se va a experimentar y
    discretización del dominio.
  \item
    Adecuar las propiedades de los flujos, tiempo de simulación,
    parámetros de salida, etc. ejemplo que más se aproxime a las
    condiciones del caso que se quiere analizar.
  \item
    Obtener una buena estabilidad numérica, garantizando la convergencia
    en elproceso iterativo. Consiguiendo, además, la independencia de la
    malla, es decir, el error numérico disminuye con el aumento del
    número de nodos,cuando las soluciones numéricas que se obtienen en
    diferentes mallados, coinciden con una tolerancia se dice que son
    independientes de la malla
    /R/\href{Ferziger\%20J.H.,\%20Perić\%20M.;\%20\%22Computational\%20Methods\%20for\%20Fluid\%20Dynamics\%22;\%203rd\%20Edition;\%20Ed.:\%20Springer;\%20New\%20York,\%202002}{Ferziger
    J.H. et al}.
  \item
    Vereficar los resultados con los obtenidos en el ensayo.
  \end{itemize}
\item
  Preparación del ensayo:

  \begin{itemize}
  \item
    Diseño de la compuerta, del sistema de apertura y de la fijación del
    pistón al canal.
  \item
    Diseño de la cámara, con una pared abierta por el fondo para dejar
    pasar el agua y una tubería unida en la parte superior. En la
    tubería se colocan diferentes diafragmas para hallar la potencía
    extraida. 
  \item
    Sensores necesarios para la adquisición de datos.
  \item
    Monitorizar y registrar las mediciones de los sensores para comparar
    los resultados con los de la simulación.
  \end{itemize}
\item
  Comparación de resultados:

  \begin{itemize}
  \item
    Es necesario destacar que los diafragmas sustituyen la turbina,
    colocada para la extracción de energía real, dado que dicho análisis
    no es objeto de este proyecto; así mismo, los flujos de aire que se
    obtendrán son demasiado pequeños y con una transitoriedad
    prácticamente nula, con lo que sería más apropiado otro tipo de
    ensayo donde se dispusiera una generación de olas continuas. 
  \item
    Así mismo, la utilización de los diafragmas, como más adelante se
    detallará, permiten obtener el caudal a partir de una medida de
    presión manométrica y un Coeficiente de Descarga. Para, de ahí,
    obtener la potencia.
  \item
    Por otro lado, de las simulaciones se obtendrán las mismas
    variables, procurando adaptar lo mejor posible las condiciones y el
    modelo del ensayo, con diferentes herramientas no comerciables, de
    código abierto y gratuitas. 
  \end{itemize}
\end{itemize}
