En este Trabajo Fin de Grado se realizará la validación de un modelo
numérico, de una ola a través de un canal, mediante simulación
computacional; y se compararán los resultados obtenidos, con un ensayo
en el laboratorio de Mecánica de Fluidos de la Escuela de Ingeniería de
Bilbao. Este modelo servirá para su aplicación al estudio de captadores
de Energía undimotriz de tipo Columna Oscilante de Agua (OWC,
Oscillating Water Column).

En general se tratará de profundizar en el área de la Mecánica de
Fluidos, y en particular, en la rama de la Dinámica de Fluidos
Computacional, la cual emplea métodos numéricos y algoritmos para
estudiar y analizar problemas que involucran fluidos en movimiento
/R/\href{Orrego,\%20S;\%20\%22Simulación\%20de\%20fluidos\%20utilizando\%20computadores:\%20una\%20moderna\%20herraamienta\%20para\%20el\%20estudio\%20y\%20análisis\%20de\%20fluidos\%22;\%20Grupo\%20de\%20Investigación\%20Mecánica\%20Aplicada;\%20Universidad\%20EAFIT,\%20Colombia;\%20noviembre\%202009}{1}.

Para ello, se hará un análisis de las tecnologías existentes para el
aprovechamiento de la Energía del Mar. Clasificándolas según la forma en
la que se puede aprovechar la energía que se presenta en el océano
/R/\href{Amundarain,\%20M.\%20(2012).\%20\%22La\%20energía\%20renovable\%20procedente\%20de\%20las\%20olas\%22;\%20Ikastorratza,\%20e-Revista\%20de\%20Didáctica\%208,\%20Retrieved\%202012/02/25\%20from\%20http://www.ehu.es/ikastorratza/8_alea/energia/energia.pdf\%20(ISSN:1988-5911)}{2},
ubicación e impacto medioambiental. Optando por el dispositivo OWC, como
el más conveniente para el estudio e instalación, en un contexto de
desarrollo sostenible.

Se comprenderán las bases y el desarrollo de la metodología empleada en
el estudio computacional de la Dinámica de Fluidos (CFD, Computational
Fluid Dynamics), describiendo la implementación y resolución de las
ecuaciones de conservación (para la masa, el momento y la energía) desde
su formulación más general hasta el caso a ilustrar. Mostrando cómo
programar los casos y qué algoritmos se emplean para la resolución más
adecuada de las ecuaciones de Navier-Stokes usando el método de
Volumenes Finitos.

El desarrollo de las simulaciones se llevará a cabo con el programa
libre OpenFOAM (Open Field Operation And Manipulation), robusto y
avanzado, ampliamente usado en la industria. Está orientado a objetos,
con estructura modular, facilitando programar nuevos resolvedores y
condiciones de entorno. Además, dispone de una amplia compatibilidad con
diversas aplicaciones. Las librerías están escritas en lenguaje C++,
dando acceso a comprender lo que ofrece un paquete comercial. Además,
dispone de una amplia comunidad de usuarios compartiendo conocimientos e
implementando nuevos modelos.

La experimentación del caso, podría hacerse en un canal o un tanque de
olas. No obstante, dada la complegidad adyacente a la generación de las
olas, se empleará el canal, adaptándo su diseño a las condiciones que se
van a simular. Es decir, se dispondrá de la estructura del canal y se
definirá una columna de agua en reposo, retenida por una compuerta.
Cuando en el instante inicial el agua colapse, se visualizará el
comportamiento de la superficie libre de líquido. Adicionalmente, para
aproximar la simulación al modelo OWC, se situará una pared al final del
recorrido para hallar la altura oscilante de agua dentro de la cámara.
También, se añadirá una tubería con un diafrgama de sección
intercambiable para calcular la máxima potencia extraida.

Finalmente, con el propósito de afianzar la destreza en la resolución de
problemas, a la hora de implementar las modificaciones a los caso que se
desee simular, se anexa información acerca de las fuentes,
procedimientos y referencias. Esto pretende servir de ayuda para
facilitar la curva de aprendizaje, a falta de normas o códigos técnicos
que guíen su desarrollo.
